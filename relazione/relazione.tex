% esempio per aggiungere un immagine
%\begin{figure}[h]
%\centering{}
%\includegraphics{img/img.png}
%\caption{uml fatto male}
%\label{img:analysis}
%\end{figure}

\documentclass[a4paper,12pt]{report}

\usepackage{alltt, fancyvrb, url}
\usepackage{graphicx}
\usepackage[utf8]{inputenc}
\usepackage{hyperref}
\usepackage[italian]{babel}
\usepackage[italian]{cleveref}

\title{Covid Simulation}

\author{Christian Ricci, Roberto Montalti, Juri Simoncini, Edoardo Savini}
\date{\today}

\begin{document}

\maketitle

\tableofcontents

\chapter{Analisi}

Il team si pone come obbiettivo lo sviluppo di una simulazione del Covid dentro l'università di Bologna. Lo scopo è poter permettere all'utente di visualizzare una possibile diffusione del Covid dentro l'università, attraverso una mappa virtuale della struttura. L'utente potrà inoltre modificare alcuni parametri della simulazione, per esempio potrà aggiungere nuove persone mentre la simulazione sta girando.

\section{Requisiti}

\subsection{Requisiti funzionali}
\begin{itemize}
\item L'applicazione permette di visualizzare una mappa dell'università, con una rappresentazione di alcune persone che si muovono all'interno della mappa.
\item Ci devono essere persone contagiate dal virus, e queste devono poter contagiare altre persone.
\item Le persone potranno indossare una mascherina che permetta a loro di abbassare la probabilità di contagio.
\item Durante la simulazione si potranno vedere quali persone sono state contagiate dal virus.
\item L'applicazione permette di iniziare una nuova simulazione con parametri differenti.
\item L'utente avrà la possibilità di girare all'interno della mappa e poter visualizzare come si diffonde il virus.
\item L'utente avrà anche la possibilità di aggiungere nuove persone alla simulazione, o di modificare alcuni parametri scelti all'inizio.
\end{itemize}

\subsection{Requisiti non funzionali}
\begin{itemize}
\item L'utente avrà la possibilità di "selezionare" una persona e poterla seguire automaticamente, semplicemente utilizzando il pulsante sinistro del mouse.
\item La simulazione dovrà avere diversi "preset" a disposizione, invece di dover scegliere le opzioni iniziali. Un preset è un insieme di opzioni iniziali per la simulazione.
\end{itemize}

\section{Analisi e modello del dominio}
Nella simulazione del Covid i soggetti principali sono le persone, le quali si muovono all'interno della mappa dell'università, ed il virus stesso, che gestisce le persone contagiate e cerca di contagiare nuove persone.

Le persone si muovono liberamente all'interno di una mappa, seguendo un apposito algoritmo di movimento. All'interno della simulazione c'è sempre almeno una persona contagiata dal virus, e c'è la possibilità che altre persone vengano contagiate se queste sono abbastanza vicine ad esso. Il meccanismo di infezione è gestito dal virus, che si occupa di guardare quali persone sono vicine ad una persona contagiata e di applicare a loro un algoritmo di infezione. Inoltre, ogni persona può indossare una mascherina, che può influenzare la possibilità di contagio.

La simulazione ha diversi parametri personalizzabili, quali il numero di persone, il numero di persone che non indossare una mascherina e il tipo di mascherina che ogni persona indossa.

Una delle principali difficoltà sarà dover trovare un algoritmo di movimento adatto alle persone, che sia realistico e non troppo banale. Un'altra difficoltà sarà nel gestire la diffusione del virus.

\chapter{Design}

\section{Architettura}

L'architettura di Covid Simulation si basa sull'MVC (Model - View - Controller) per quanto riguarda la parte della GUI (schermata iniziale, schermata di simulazione, schermata di pausa), e la visione ad alto livello del programma ma dentro alla simulazione stessa (dentro al Model) è stato usato un pattern differente, che si basa su un architettura a componenti seguita da alcuni videogiochi.
La classe Main funge da controller delle schermate: ad ogni aggiornamento di frame esso guarda in quale schermata è l'applicazione e va a chiamare un apposito metodo che per l'aggiornamento delle proprietà delle schermate.

Un apposito controller per la GUI, chiamato StartScreenController, gestisce le schermate iniziale e di pausa. Questo controller riesce a comunicare con il Main attraverso alcuni callback, e comunica con il controller della simulazione mandandogli direttamente i messaggi.

La view per la GUI non è stata fatto usando codice, bensì abbiamo utilizzato una libreria chiamata Nifty GUI, che permette lo sviluppo di GUI utilizzando XML. I bottoni della view sono solo legati ai metodi del controller: in questo modo, se volessimo cambiare la View, ci basterebbe assicurarci di chiamare i relativi metodi del controller per ogni bottone.

Il controller della simulazione è chiamato, appunto, Simulation ed esso gestisce il Model della simulazione, che comprende concetti quale il Virus e le persone. Esso ha disponibile un metodo per aggiornare tutte le persone, ed altri metodi per gestirle.

Ogni persona fa utilizzo di componenti, che gestiscono un solo concetto o sottosistema dell'engine: ad esempio, il MovementComponent gestisce il movimento della persona, mentre il GraphicsComponent gestisce tutto ciò che riguarda la grafica.

In figura è mostrata l'architettura ad alto livello dell'applicazione.

% qui ci vuole un bel uml immerdato

\section{Design dettagliato}

\subsection{Christian Ricci}

Io mi sono occupato per lo più di alcuni componenti del Model: Ad esempio, la classe della persona e alcuni dei suoi componenti. In particolare, io ho subito pensato, all'inizio del progetto, di creare un architettura a componenti per la persona, dato che sapevo che altrimenti la classe sarebbe diventata molto complessa, toccando varie parti dell'engine. Con un architettura a componenti, si va a scorporare queste parti, creando multiple classi che però diventano molto semplici, occupandosi di una sola cosa.

Assieme a Roberto Montalti ed Edoardo Savini ho creato il Movement Component e l'algoritmo di movimento: per l'algoritmo abbiamo pensato che la persona dovesse scegliere un punto casuale dentro la mappa, e grazie ad una libreria per l'AI offerta dal JMonkeyEngine la persona avrebbe trovato la strada migliore per arrivarci. Il Movement Component in sè deve prima richiedere la strada e poi iniziare a seguirla; la richiesta di una nuova strada da seguire può però essere molto costosa, dato che l'algoritmo per trovare la strada migliore è dispendioso. Questo si traduce in una serie di stati per il componente, e quindi abbiamo pensato di usare un pattern chiamato "State Pattern" per gestire tutti questi stati. Inoltre, abbiamo dovuto astrarre la richiesta della strada in una nuova classe, che facesse uso di una thread pool per gestire tutte le richieste di una nuova strada per tutte le persone.

In figura è mostrato un diagramma UML per il Movement Component.

Ho inoltre creato una prima versione della simulazione, che aveva semplicemente 2 metodi per inizializzarla e aggiornarla. Juri Simoncini ha poi esteso la simulazione, creando alcuni metodi per manipolare le persone.

Ho anche creato cose non legate al Model dell'applicazione: ad esempio ho creato una semplice gestione degli stati delle schermate nella classe Main, e ho creato un'interfaccia di callback per lasciare che il controller della GUI riesca a comunicare con la classe Main senza dover chiamare i suoi metodi direttamente: così facendo, si riesce a scorporare il controller della GUI dal resto dell'applicazione.

\subsection{Roberto Montalti}

\subsection{Juri Simoncini}
Io mi sono occupato dell'algortimo di infezione, la challenge principale erano capire su quali aspetti basarci per l'infezione, 
ho preferito utilizzare come parametri dell'algoritmo le proprietà della mascherina avendo cosi un'algoritmo semplice e leggero.
In seguito mi sono occupato del lato GUI: lato grafico e lato di controller.
La scelta principale da fare era decidere se suddividere i controller per schermata oppure tenerne uno per tutte le schermate.
Ho deciso di optare per la seconda soluzione dato che le feature da implementare non erano in grande quantità.
Però per alleggerire il codice ho deciso di suddividire le implementazioni dei metodi in varie classi suddivise per schermate.
Ho poi lavorato sulla simulazione per la manipolazione delle persone dato che mi era utile per le funzionalità del controller.



\subsection{Edoardo Savini}

\chapter{Sviluppo}

\section{Testing automatizzato}

% questo lo fa savini

\section{Metodologia di lavoro}

\subsection{Metodologia in generale}

Abbiamo deciso di iniziare a sviluppare questo progetto con una fase di analisi, per capire esattamente cosa bisognasse fare, come potevamo suddividerci il lavoro e quali librerie potevamo utilizzare. Abbiamo presentato tutti le nostre idee riguardo al Model dell'applicazione. In retrospettiva la nostra fase di analisi potrebbe essere stata troppo corta, dato che ci siamo accorti più o meno a metà del progetto che l'engine non era particolarmente buono e abbiamo faticato molto per risolvere alcuni problemi legati alla librerie scelte.

Abbiamo deciso di suddividere in modo equo il lavoro, ma abbiamo anche che sarebbe stato meglio collaborare insieme su alcuni componenti dell'applicazione che sono risultati più difficili da scrivere, ad esempio l'algoritmo di movimento della persona.

\subsection{Distributed Version Control System}
Per il DVCS è stato utilizzato git, dato che tutti i membri del team avevano esperienza solo su questo. Abbiamo creato una repository su github e abbiamo deciso di creare un branch per ogni membro del team. Questo alla fine ci è sembrato poco produttivo, data la difficoltà per ognuno di stare alla pari con le feature che altri membri creavano.

\subsection{Christian Ricci}

Sviluppo del Model dell'applicazione, in particolare la classe Simulazione, la classe Persona e alcuni dei suoi componenti e la classe Spawner. Ho inoltre contribuito ad aggiustare varie classi, tra cui il Virus e il controller della GUI.
Classi sviluppate singolarmente:

\subsection{Roberto Montalti}

\subsection{Juri Simoncini}
Lato GUI sia a lato grafico che a lato logico utilizzando la libreria nifty-gui.
L'implementazione grafica si trova dentro al file screen.xml che poi viene passato allo StartScreenController, il controller della nostra GUI.
Tutta l'interfaccia è stata organizzata per schermate, quindi è presente per ogni schermata (Start, Commands, Edit, Pause) uno screen nel file screen.xml.
Lo StartScreenController presenta al suo interno tutti i metodi che vengono chiamati dai vari screen (bottoni, pulsanti, textField...).
Per l'edit screen ho implementato una classe chiamata EditComponent con all'interno la logica per i metodi utili allo StartScreenController, stessa cosa ho attuato per il load screen con il loadComponent.
Mentre per il testo presente nell' HUD ho optato per un oggetto separato dal controller, per non sovraccaricarne il lavoro, chiamato HudText.
Il testo presente così riesce ad aggiornarsi in tempo reale grazie al metodo SimpleUpdate presente nel Main.
L'algoritmo di infezione che viene utilizzato per infettare le varie persone presenti nella mappa.
L'algoritmo si basa sulle dell'oggeto mask: Protection (FP1, FP2, FP3) e Status (UP, DOWN).
Ho preferito assegnare più rilevanza allo status in base anche alle esperienze nella vita reale dato che una mascherina abbassata avrà un effetto maggiore sulla trasmissione del virus.
Quindi lo status in percentuale ha un 70\% di rilevanza mentre la protection il restante 30.
Perciò ho pensato alla trasmissibilità dell'infezione come una scala di valori da 0 a 100 in cui, per esempio, se la mascherina è abbassata ed è di tipo FP1 le probabilità di trasmissione è pari a 100.
Per rappresentare la randomicità del contagio viene estratto un valore random da 0 a 100 e se questo è minore del valore della trasmissibilità calcolato la persona viene contagiata, quindi maggiore è il valore di trasmissibilità maggiori sono le porbabilità di essere contagiato.

\subsection{Edoardo Savini}

\section{Note di sviluppo}

\subsection{Christian Ricci}

\begin{itemize}
\item JMonkeyEngine: engine utilizzato per tutta l'applicazione.
\item JMonkeyEngine AI: libreria utilizzata all'interno dell'implementazione del Movement Component.
\item Thread Pool: un uso di una thread pool è stato fatto dentro al Movement Component per gestire le richieste di nuove strade.
\item Generici: l'interfaccia di callback dentro il GUI controller è un'interfaccia funzionale generica, per permettere di registrare un qualunque callback.
\end{itemize}

\subsection{Roberto Montalti}

\subsection{Juri Simoncini}
\item JMonkeyEngine: engine utilizzato per tutta l'applicazione.
\item Nifty-gui: libreria utilizzata in tutto lo sviluppo della gui.
\item Uso di lambda expressions.

\subsection{Edoardo Savini}

\chapter{Commenti finali}

\section{Autovalutazione e lavori futuri}
\subsection{Christian Ricci}

\subsection{Roberto Montalti}

\subsection{Juri Simoncini}
Avrei potuto sicuramente fare di più ma per vari problemi non sono riuscito a mettere in atto tutti gli insegnamenti compresi a lezione soprattutto per quanto riguarda l'utilzzo del pattern MVC nella gui che sarebbe potuto essere attuato in maniera migliore.
D'altro canto ritengo però di averci messo tutto l'impegno possibile per aiutare i miei colleghi nello sviluppo dell'applicazione e nello sviluppo portato avanti da me.
In conclusione ritengo che lo sviluppo di questa applicazione abbia ampliato le mie conoscenze facendomi conoscere un engine tutto nuovo e perciò avendo acquisito una nuova skill. 

\subsection{Edoardo Savini}



\section{Difficoltà incontrate e commenti per i docenti}


\appendix
\chapter{Guida utente}

\chapter{Esercitazioni di laboratorio}

% nessuno li ha fatti vero?

\bibliographystyle{alpha}
\bibliography{13-template}

\end{document}
